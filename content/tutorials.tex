\newcommand{\tutorialauthors}[1]{
    \paperauthor{\citefullauthornames{#1}}}
\chapter{\label{chap:Tutorials}Tutorials:Tuesday, November 1}
\thispagestyle{emptyheader}
\setheaders{Tutorials}{\daydateyear}
\setlength{\parindent}{0in}
\setlength{\parskip}{2ex}
\renewcommand{\baselinestretch}{0.87}
\section*{Overview}
\renewcommand{\arraystretch}{1.2}
%begin{longtable}{cc}
%begin{longtable}{r>{\raggedright}p{0.83\columnwidth}}
% & \textbf{\large{}September 17\smallskip{}} \tabularnewline
%09:00 -- 12:30 & \textbf{Morning Sessions}\tabularnewline
% & \linktutorial{Tutorial-001}{\TutLocA}
%
%\linktutorial{Tutorial-002}{\TutLocB}
%\linktutorial{Tutorial-003}{\TutLocC}\tabularnewline
%
%2:30 -- 14:00 & \textbf{Lunch break}\tabularnewline
%14:00 -- 17:30 & \textbf{Afternoon Sessions}\tabularnewline
%& \linktutorial{Tutorial-004}{\TutLoc4}
%\linktutorial{Tutorial-005}{\TutLoc5}
%\linktutorial{Tutorial-006}{\TutLoc6}
%\end{longtable}

%\clearpage{}

%\begin{tabular}{lllp{0.7\textwidth}r}
%9:00 & -- & 12:30 & {\bfseries Morning Tutorials} & \hfill \\ 
%& & & \papertitle{Tutorial-001}\hfill & \emph{\TutLocA} \hfill \\ 
% & & & \tutorialauthors{Tutorial-001} & \hfill \\
% & & & \papertitle{Tutorial-002}\hfill & \emph{\TutLocA} \hfill \\ 
% & & & \tutorialauthors{Tutorial-002} & \hfill \\
% & & & \papertitle{Tutorial-003}\hfill & \emph{\TutLocA} \hfill \\ 
% & & & \tutorialauthors{Tutorial-003} & \hfill \\
% & & & \papertitle{Tutorial-004}\hfill & \emph{\TutLocA} \hfill \\ 
% & & & \tutorialauthors{Tutorial-004} & \hfill \\
% & & & \papertitle{Tutorial-005}\hfill & \emph{\TutLocA} \hfill \\ 
% & & & \tutorialauthors{Tutorial-005} & \hfill \\
% & & & \papertitle{Tutorial-006}\hfill & \emph{\TutLocA} \hfill \\ 
% & & & \tutorialauthors{Tutorial-006} & \hfill \\
%\end{tabular}



\begin{SingleTrackSchedule}
9:00 & -- & 12:30 & {\bfseries Morning Tutorials} \hfill \\
  & & & \papertitle{Tutorial-001}\newline
  \tutorialauthors{Tutorial-001} \hfill\emph{\TutLocA} \\
  & & & \papertitle{Tutorial-002} \newline
  \tutorialauthors{Tutorial-002} \hfill\emph{\TutLocB} \\
   & & & \papertitle{Tutorial-003} \newline
  \tutorialauthors{Tutorial-003} \hfill\emph{\TutLocC} \\
12:30 & -- & 14:00 & {\bfseries Lunch break} \hfill \\
14:00 & -- & 17:30 & {\bfseries Afternoon Tutorials} \hfill \\
 & & & \papertitle{Tutorial-004}\newline
  \tutorialauthors{Tutorial-004} \hfill\emph{\TutLocD} \\
  & & & \papertitle{Tutorial-005}\newline
  \tutorialauthors{Tutorial-005} \hfill\emph{\TutLocE} \\
   & & & \papertitle{Tutorial-006}\newline
  \tutorialauthors{Tutorial-006} \hfill\emph{\TutLocF} \\
\end{SingleTrackSchedule}

\clearpage{}

\begin{bio}
\textbf{Chris Dyer (CMU)}
 Chris’s research looks at the intersection between machine learning and natural language processing, in particular focusing on representation learning, structured models, and multilinguality. He has received best paper nominations and awards at NAACL, EMNLP, and ACL. He is the author of the CNN toolkit, which was used by the Jelinek 2016 Summer Workshop on neural machine translation.

\textbf{Yoav Goldberg (Bar Ilan)}
 Yoav’s research focuses around structure prediction problems in natural language, in particular prediction of syntactic representations, greedy methods, semi-supervised learning and robust cross-domain performance. Recently, Yoav became involved in the application of neural-network based models to NLP problems, with two core focuses: better understanding the neural network building blocks; and the use of neural networks for structured problems. Yoav wrote a primer on neural-networks method in NLP, and is responsible for the python bindings of the CNN library.

\textbf{Graham Neubig (NAIST)}
 Graham’s research focuses on machine learning methods for language and speech processing. His work puts a particular focus on machine translation and speech translation, but also covers other NLP tasks such as tagging and syntactic/semantic parsing, and methods for learning from unlabeled data such as unsupervised, semi-supervised, or active learning. He is interested in neural network models that incorporate our linguistic intuitions, and is one of the core developers of the CNN toolkit. 
\index{Dyer, Chris} \index{Goldberg, Yoav} \index{Neubig, Graham} 
\end{bio}


\begin{tutorial}{Tutorial-001}{Tuesday, November 1 -- 09:00--12:30}
{\TutLocA}

 This tutorial aims to bring NLP researchers up to speed with the current techniques in deep learning and neural networks, and show them how they can turn their ideas into practical implementations. We will start with simple classification models (logistic regression and multilayer perceptrons) and cover more advanced patterns that come up in NLP such as recurrent networks for sequence tagging and prediction problems, structured networks (e.g., compositional architectures based on syntax trees), structured output spaces (sequences and trees), attention for sequence-to-sequence transduction, and feature induction for complex algorithm states. A particular emphasis will be on learning to represent complex objects as recursive compositions of simpler objects. This representation will reflect characterize standard objects in NLP, such as the composition of characters and morphemes into words, and words into sentences and documents. In addition, new opportunities such as learning to embed ``algorithm states" such as those used in transition-based parsing and other sequential structured prediction models (for which effective features may be difficult to engineer by hand) will be covered.

Everything in the tutorial will be grounded in code —- we will show how to program seemingly complex neural-net models using toolkits based on the computation-graph formalism. Computation graphs decompose complex computations into a DAG, with nodes representing inputs, target outputs, parameters, or (sub)differentiable functions (e.g., ``tanh", ``matrix multiply", and ``softmax"), and edges represent data dependencies. These graphs can be run “forward” to make predictions and compute errors (e.g., log loss, squared error) and then ``backward" to compute derivatives with respect to model parameters. In particular we'll cover the Python bindings of the CNN library. CNN has been designed from the ground up for NLP applications, dynamically structured NNs, rapid prototyping, and a transparent data and execution model. 

\end{tutorial} 

\clearpage{}

\begin{bio}
	\textbf{Deepak Venugopal} completed his Ph.D. at the University of Texas at Dallas in 2015 after which he joined the department of computer science at the University of Memphis where he is currently an assistant professor. His main research interests are in Probabilistic Graphical Models, Statistical Relational Models and their applications in natural language processing and cyber-security. His re- search work has resulted in several key techniques for scalable inference and learning in Statistical Relational Models.

\textbf{Vibhav Gogate} is an Assistant Professor in the Computer Science Department at the Univer- sity of Texas at Dallas. He got his Ph.D. at University of California, Irvine in 2009 and then did a two-year post-doc at University of Washington. His current research focus is on probabilistic graphical models and its first-order logic based extensions such as Markov logic. He is best known for his work on inference algorithms that combine the power of logic and probability including lifted probabilistic inference algorithms. He is the co-winner of the last two probabilistic inference competitions the 2010 UAI approximate inference challenge and the 2012 PASCAL probabilistic inference competition.

\textbf{Vincent Ng} is an Associate Professor of Computer Science and a member of the Human Lan- guage Technology Research Institute at the University of Texas at Dallas. He is best known for his work on coreference resolution in the NLP community. He has recently been collaborating with the co-authors of this proposal on modeling complex NLP tasks using Markov Logic. 
\index{Venugopal, Deepak} \index{Gogate, Vibhav} \index{Ng, Vincent}

\end{bio} 

\begin{tutorial}{Tutorial-002}{Tuesday, November 1 -- 09:00--12:30}
{\TutLocB}
	In the early days of the statistical NLP era, many language processing tasks were tackled using the so-called pipeline architecture: the given task is broken into a series of sub-tasks such that the output of one sub-task is an input to the next sub-task in the sequence. The pipeline architecture is appealing for various reasons, including modularity, modeling convenience, and manageable computational complexity. However, it suffers from the error propagation problem: errors made in one sub-task are propagated to the next sub-task in the sequence, leading to poor accuracy on that sub-task, which in turn leads to more errors downstream. Another disadvantage associated with it is lack of feedback: errors made in a sub-task are often not corrected using knowledge uncovered while solving another sub-task down the pipeline.

Realizing these weaknesses, researchers have turned to joint inference approaches in recent years. One such approach involves the use of Markov logic, which is defined as a set of weighted first-order logic formulas and, at a high level, unifies first-order logic with probabilistic graphical models. It is an ideal modeling language (knowledge representation) for compactly representing relational and uncertain knowledge in NLP. In a typical use case of MLNs in NLP, the application designer describes the background knowledge using a few first-order logic sentences and then uses software packages such as Alchemy, Tuffy , and Markov the beast to perform learning and inference (prediction) over the MLN. However, despite its obvious advantages, over the years, researchers and practitioners have found it difficult to use MLNs effectively in many NLP applications. The main reason for this is that it is hard to scale inference and learning algorithms for MLNs to large datasets and complex models, that are typical in NLP.

In this tutorial, we will introduce the audience to recent advances in scaling up inference and learning in MLNs as well as new approaches to make MLNs a “black-box” for NLP applications (with only minor tuning required on the part of the user). Specifically, we will introduce attendees to a key idea that has emerged in the MLN research community over the last few years, lifted inference , which refers to inference techniques that take advantage of symmetries (e.g., synonyms), both exact and approximate, in the MLN. We will describe how these next-generation inference techniques can be used to perform effective joint inference. We will also present our new software package for inference and learning in MLNs, Alchemy 2.0, which is based on lifted inference, focusing primarily on how it can be used to scale up inference and learning in large models and datasets for applications such as semantic similarity determination, information extraction and question answering. 
\end{tutorial} 

\clearpage{}


\begin{bio}
\textbf{Zhiyuan Chen, Google}
 Zhiyuan Chen completed his Ph.D. at the University of Illinois at Chicago (UIC) and joined Google in 2016. His research interests include Text Mining, Machine Learning, and Statistical Natural Language Processing. He has designed several lifelong learning algorithms to automatically mine valuable information from text documents. He has published more than 15 full research papers at premier conferences.

\textbf{Bing Liu, University of Illinois at Chicago (UIC)}
 Bing Liu is a professor of Computer Science at the University of Illinois at Chicago (UIC). He received his PhD in Artificial Intelligence from the University of Edinburgh. His current research interests include sentiment analysis, NLP, data mining, and machine learning. He has published extensively in top conferences and journals, and also three books. Two of his papers received test-of-time awards from KDD. He is a Fellow of ACM, AAAI, and IEEE. 
 \index{Chen, Zhiyuan} \index{Liu, Bing} 
\end{bio}
\begin{tutorial}{Tutorial-003}{Tuesday, November 1 -- 09:00--12:30} {\TutLocC}

Machine learning (ML) has been successfully used as a prevalent approach to solving numerous NLP problems. However, the classic ML paradigm learns in isolation. That is, given a dataset, an ML algorithm is executed on the dataset to produce a model without using any related or prior knowledge. Although this type of isolated learning is very useful, it also has serious limitations as it does not accumulate knowledge learned in the past and use the knowledge to help future learning, which is the hallmark of human learning and human intelligence. Lifelong machine learning (LML) aims to achieve this capability. Specifically, it aims to design and develop computational learning systems and algorithms that learn as humans do, i.e., retaining the results learned in the past, abstracting knowledge from them, and using the knowledge to help future learning. In this tutorial, we will introduce the existing research of LML and to show that LML is very suitable for NLP tasks and has potential to help NLP make major progresses. 

\end{tutorial} 

\clearpage{}









\begin{bio}
\textbf{ Yue Zhang, Assistant Professor, Singapore University of Technology and Design}
 Yue Zhang is an assistant professor at Singapore University of Technology and Design (SUTD). Before joining SUTD, he worked as a postdoctoral research fellow at University of Cambridge. He received his DPhil and MSc degrees from University of Oxford, and BEng from Tsinghua University. His main research interest includes machine learning, syntactic parsing and text mining. Over the recent years, he worked intensively on deep learning and sentiment analysis. He serves in the standing review committee of TACL, and was area chair of COLING'14, NAACL'15 and EMNLP'15. Yue Zhang has given a tutorial at NAACL'10 and a tutorial at ACL'14.

\textbf{Duy Tin Vo, PhD Candidate, Singapore University of Technology and Design}
 Duy Tin Vo is a PhD candidate at Singapore University of Technology and Design (SUTD) under the supervision of Yue Zhang. He received his BEng from Cantho University. His current research focuses on applying machine learning and deep learning in sentiment analysis. 
 \index{Zhang, Yue} \index{Tin Vo, Duy}
\end{bio}
\begin{tutorial}{Tutorial-004}{Tuesday, November 1 -- 14:00--17:30}{\TutLocD}
 Sentiment analysis has been a major research topic in natural language processing (NLP). Traditionally, the problem has been attacked using discrete models and manually-defined sparse features. Over the past few years, neural network models have received increased research efforts in most sub areas of sentiment analysis, giving highly promising results. A main reason is the capability of neural models to automatically learn dense features that capture subtle semantic information over words, sentences and documents, which are difficult to model using traditional discrete features based on words and ngram patterns. This tutorial gives an introduction to neural network models for sentiment analysis, discussing the mathematics of word embeddings, sequence models and tree structured models and their use in sentiment analysis on the word, sentence and document levels, and fine-grained sentiment analysis. The tutorial covers a range of neural network models (e.g. CNN, RNN, RecNN, LSTM) and their extensions, which are employed in four main subtasks of sentiment analysis:
\begin{itemize}
\item     Sentiment-oriented embeddings;
\item      Sentence-level sentiment;
\item      Document-level sentiment;
\item      Fine-grained sentiment. 
\end{itemize}
The content of the tutorial is divided into 3 sections of 1 hour each. We assume that the audience is familiar with linear algebra and basic neural network structures, introduce the mathematical details of the most typical models. First, we will introduce the sentiment analysis task, basic concepts related to neural network models for sentiment analysis, and show detail approaches to integrate sentiment information into embeddings. Sentence-level models will be described in the second section. Finally, we will discuss neural network models use for document-level and fine-grained sentiment. 
\end{tutorial} 

\clearpage{}









\begin{bio}
\textbf{ Rafael E Banchs, Researcher, Institute for Infocomm Research, Singapore}
 Rafael is a Research Scientist at the Institute for Infocomm Research, in Singapore, where he leads the Dialogue Technology Lab of the Human Language Technology Department. His main area of research is focused on the construction and use of semantic representations to support different natural language processing applications, including machine translation, information retrieval, natural language understanding and chat-oriented dialogue. 
 \index{E Banchs, Rafael}
\end{bio}
\begin{tutorial}{Tutorial-005}{Tuesday, November 1 -- 14:00--17:30} {\TutLocE}
 The mathematical metaphor offered by the geometric concept of distance in vector spaces with respect to semantics and meaning has been proven to be useful in many monolingual natural language processing applications. There is also some recent and strong evidence that this paradigm can also be useful in the cross-language setting. In this tutorial, we present and discuss some of the most recent advances on exploiting the vector space model paradigm in specific cross-language natural language processing applications, along with a comprehensive review of the theoretical background behind them.

First, the tutorial introduces some fundamental concepts of distributional semantics and vector space models. More specifically, the concepts of distributional hypothesis and term-document matrices are revised, followed by a brief discussion on linear and non-linear dimensionality reduction techniques and their implications to the parallel distributed approach to semantic cognition. Next, some classical examples of using vector space models in monolingual natural language processing applications are presented. Specific examples in the areas of information retrieval, related term identification and semantic compositionality are described.

Then, the tutorial focuses its attention on the use of the vector space model paradigm in cross-language applications. To this end, some recent examples are presented and discussed in detail, addressing the specific problems of cross-language information retrieval, cross-language sentence matching, and machine translation. Some of the most recent developments in the area of Neural Machine Translation are also discussed.

Finally, the tutorial concludes with a discussion about current and future research problems related to the use of vector space models in cross-language settings. Future avenues for scientific research are described, with major emphasis on the extension from vector and matrix representations to tensors, as well as the problem of encoding word position information into the vector-based representations. 
\end{tutorial} 

\clearpage{}







\begin{bio}
\textbf{ Xu Sun, Associate Professor, Peking University} 
 Xu Sun is currently an Associate Professor at the School of EECS, Peking University. He got PhD from The University of Tokyo in 2010. He worked at The University of Tokyo, Cornell University, and The Hong Kong Polytechnic University as research fellow/associate from 2010 to 2012. His research interests focus on natural language processing and machine learning, especially on structured natural language processing and structured prediction, and with related publications on ACL, NIPS, IJCAI, Comput. Linguist., IEEE T-KDE, etc. He has been area chair of EMNLP; program committee member of ACL, IJCAI, AAAI, NAACL; journal reviewer of IEEE T-PAMI, Comput. Linguist., IEEE T-ASL, and so on.

\textbf{Yansong Feng, Lecturer, Peking University}
 Yansong Feng is a Lecturer at the Institute of Computer Science and Technology, Peking University. He received his Ph.D. degree from ILCC, the University of Edinburgh and worked as a postdoctoral research assistant there in 2011. His research interests focus on applying machine learning techniques to natural language processing and information retrieval applications, such as information extraction, question answering, knowledge acquisition, and so on. His works have been published in top refereed international journals and conferences, such as IEEE TPAMI, ACL, VLDB, EMNLP, NAACL, AAAI, etc. He has served as program committee member of AAAI, EMNLP, NAACL, WWW; journal reviewer of IEEE TASLP, IEEE TKDE, TACL, and so on. 
 \index{Sun, Xu} \index{Feng, Yansong}
\end{bio}
\begin{tutorial}{Tutorial-006}{Tuesday, November 1 -- 14:00--17:30} {\TutLocF}
 Many important NLP tasks are casted as structured prediction problems, and try to predict certain forms of structured output from the input. Examples of structured prediction include POS tagging, named entity recognition, PCFG parsing, dependency parsing, machine translation, and many others. When apply structured prediction to a specific NLP task, there are the following challenges:
\begin{enumerate}
\item  Model selection: Among various models/algorithms with different characteristics, which one should we choose for a specific NLP task?
\item    Training: How to train the model parameters effectively and efficiently?
\item    Overfitting: To achieve good accuracy on test data, it is important to control the overfitting from the training data. How to control the overfitting risk for structured prediction?
\end{enumerate}
This tutorial will provide a clear overview of recent advances in structured prediction methods and theories, and address the above issues when we apply structured prediction to NLP tasks. We will introduce large margin methods (e.g., perceptrons, MIRA), graphical models (e.g., CRFs), and deep learning methods (e.g., RNN, LSTM), and show the respective advantages and disadvantages for NLP applications. For the training algorithms, we will introduce online/ stochastic training methods, and we will introduce parallel online/stochastic learning algorithms and theories to speed up the training (e.g., the Hogwild algorithm). For controlling the overfitting from training data, we will introduce the weight regularization methods, structure regularization, and implicit regularization methods. 
\end{tutorial} 

\clearpage{}






%\begin{bio}
%
%\end{bio}
%\begin{tutorial}{Tutorial-003}{Tuesday, November 1 -- 09:00--12:30}
%{\TutLocC}
%
%\end{tutorial} 
%
%\clearpage{}





