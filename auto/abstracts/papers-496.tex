When humans read text, they fixate some words and skip others. However, there have been few attempts to explain skipping behavior with computational models, as most existing work has focused on predicting reading times (e.g.,{\textasciitilde}using surprisal). In this paper, we propose a novel approach that models both skipping and reading, using an unsupervised architecture that combines a neural attention with autoencoding, trained on raw text using reinforcement learning. Our model explains human reading behavior as a tradeoff between precision of language understanding (encoding the input accurately) and economy of attention (fixating as few words as possible). We evaluate the model on the Dundee eye-tracking corpus, showing that it accurately predicts skipping behavior and reading times, is competitive with surprisal, and captures known qualitative features of human reading.
