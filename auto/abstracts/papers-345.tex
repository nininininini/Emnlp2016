The state-of-the-art named entity recognition (NER) systems are statistical machine learning models that have strong generalization capability (i.e., can recognize unseen entities that do not appear in training data) based on lexical and contextual information. However, such a model could still make mistakes if its features favor a wrong entity type. In this paper, we utilize Wikipedia as an open knowledge base to improve multilingual NER systems. Central to our approach is the construction of high-accuracy, high-coverage multilingual Wikipedia entity type mappings. These mappings are built from weakly annotated data and can be extended to new languages with no human annotation or language-dependent knowledge involved. Based on these mappings, we develop several approaches to improve an NER system. We evaluate the performance of the approaches via experiments on NER systems trained for 6 languages. Experimental results show that the proposed approaches are effective in improving the accuracy of such systems on unseen entities, especially when a system is applied to a new domain or it is trained with little training data (up to 18.3 \$F\_1\$ score improvement).
