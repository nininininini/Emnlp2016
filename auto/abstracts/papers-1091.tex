Linguistic research on multilingual societies has indicated that there is usually a preferred language for expression of emotion and sentiment. Paucity of data has limited such studies to participant interviews and speech transcriptions from small groups of speakers. In this paper, we report a study on 430,000 unique tweets from Indian users, specifically Hindi-English bilinguals, to understand the language of preference, if any, for expressing opinion and sentiment. To this end, we develop classifiers for opinion detection in these languages, and further classifying opinionated tweets into positive, negative and neutral sentiments. Our study indicates that Hindi (i.e., the native language) is preferred over English for expression of negative opinion and swearing. As an aside, we explore some common pragmatic functions of code-switching through sentiment detection.
