Text reuse refers to citing, copying or alluding text excerpts from a text resource to a new context. While detecting reuse in contemporary languages is well supported—given extensive research, techniques, and corpora— automatically detecting historical text reuse is much more difficult. Corpora of historical languages are less documented and often encompass various genres, linguistic varieties, and topics. In fact, historical text reuse detection is much less understood and empirical studies are necessary to enable and improve its automation. We present a linguistic analysis of text reuse in two ancient data sets. We contribute an automated approach to analyze how an original text was transformed into its reuse, taking linguistic resources into account to understand how they help characterizing the transformation. It is complemented by a manual analysis of a subset of the reuse. Our results show the limitations of approaches focusing on literal reuse detection. Yet, linguistic resources can effectively support understanding the non-literal text reuse transformation process. Our results support practitioners and researchers working on understanding and detecting historical reuse.
