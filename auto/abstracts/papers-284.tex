The problem of noisy and unbalanced training data for supervised keyphrase extraction results from the subjectivity of keyphrase assignment, which we quantify by crowdsourcing keyphrases for news and fashion magazine articles with many annotators per document. We show that annotators exhibit substantial disagreement, meaning that single annotator data could lead to very different training sets for supervised keyphrase extractors. Thus, annotations from single authors or readers lead to noisy training data and poor extraction performance of the resulting supervised extractor. We provide a simple but effective solution to still work with such data by reweighting the importance of unlabeled candidate phrases in a two stage Positive Unlabeled Learning setting. We show that performance of trained keyphrase extractors approximates a classifier trained on articles labeled by multiple annotators, leading to higher average F1 scores and better rankings of keyphrases. We apply this strategy to a variety of test collections from different backgrounds and show improvements over strong baseline models.
