Sluicing is an elliptical process where the majority of a question can go unpronounced as long as there is a salient antecedent in previous discourse. This paper considers the task of antecedent selection: finding the correct antecedent for a given case of sluicing. We argue that both syntactic and discourse relationships are important in antecedent selection, and we construct linguistically sophisticated features that describe the relevant relationships. We also define features that describe the relation of the content of the antecedent and the sluice type. We develop a linear model which achieves accuracy of 72.4\%, a substantial improvement over a strong manually constructed baseline. Feature analysis confirms that both syntactic and discourse features are important in antecedent selection.
