In this work, we study parameter tuning towards the M2 metric, the standard metric for automatic grammar error correction (GEC) tasks. After implementing M2 as a scorer in the Moses tuning framework, we investigate interactions of dense and sparse features, different optimizers, and tuning strategies for the CoNLL-2014 shared task. We notice erratic behavior when optimizing sparse feature weights with M2 and offer partial solutions. We find that a bare-bones phrase-based SMT setup with task-specific parameter-tuning outperforms all previously published results for the CoNLL-2014 test set by a large margin (46.37\% M2 over previously 41.75\%, an SMT system with neural features) while being trained on the same, publicly available data. Our newly introduced dense and sparse features widen that gap, and we improve the state-of-the-art to 49.49\% M2.
