We present a study on two key characteristics of human syntactic annotations: anchoring and agreement. Anchoring is a well known cognitive bias in human decision making, where judgments are drawn towards pre-existing values. We study the influence of anchoring on a standard approach to creation of syntactic resources where syntactic annotations are obtained via human editing of tagger and parser output. Our experiments demonstrate a clear anchoring effect and reveal unwanted consequences, including overestimation of parsing performance and lower quality of annotations in comparison with human-based annotations. Using sentences from the Penn Treebank WSJ, we also report systematically obtained inter-annotator agreement estimates for English dependency parsing. Our agreement results control for parser bias, and are consequential in that they are on par with state of the art parsing performance for English newswire. We discuss the impact of our findings on strategies for future annotation efforts and parser evaluations.
