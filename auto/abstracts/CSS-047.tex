Hate speech in the form of racism and sexism is commonplace on the internet (Waseem and Hovy, 2016). For this reason, there has been both an academic and an industry interest in detection of hate speech. The volume of data to be reviewed for creating data sets encour- ages a use of crowd sourcing for the annotation efforts. In this paper, we provide an examination of the influence of annotator knowledge of hate speech on classification models by comparing classification results obtained from training on expert and amateur annotations. We provide an evaluation on our own data set and run our models on the data set released by Waseem and Hovy (2016). We find that amateur annotators are more likely than expert annotators to label items as hate speech, and that systems trained on ex- pert annotations outperform systems trained on amateur annotations.
