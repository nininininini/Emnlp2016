We address the problem of Part of Speech tagging (POS) in the context of linguistic code switching (CS). CS is the phenomenon where a speaker switches between two languages or variants of the same language within or across utterances, known as intra-sentential or inter-sentential CS, respectively. Processing CS data is especially challenging in intra-sentential data given state of the art monolingual NLP technology since such technology is geared toward the processing of one language at a time. In this paper we explore multiple strategies of applying state of the art POS taggers to CS data. We investigate the landscape in two CS language pairs, Spanish-English and Modern Standard Arabic and Arabic dialects. We compare the use of two POS taggers vs. a unified tagger trained on CS data. Our results show that applying the ML framework using two state of the art POS taggers achieves better performance compared to all other approaches that we adopt in our study.
