In this paper we are introducing work in progress towards the development of an infrastructure (i.e., design, methodology, creation and description) of linguistic and extra-linguistic data samples acquired from people diagnosed with subjective or mild cognitive impairment and healthy, age-matched controls. The data we are currently collecting consists of various types of modalities; i.e. audio-recorded spoken language samples; transcripts of the audio recordings (text) and eye tracking measurements. The integration of the extra-linguistic information with the linguistic phenotypes and measurements elicited from audio and text, will be used to extract, evaluate and model features to be used in machine learning experiments. In these experiments, classification models that will be trained, that will be able to learn from the whole or a subset of the data to make predictions on new data in order to test how well a differentiation between the aforementioned groups can be made. Features will be also correlated with measured outcomes from e.g. language-related scores, such as word fluency, in order to investigate whether there are relationships between various variables.
