The advent of social media and its prosperity enable users to share their opinions and views. Understanding users' emotional states might provide the potential to create new business opportunities. Automatically identifying users' emotional states from their texts and classifying emotions into finite categories such as joy, anger, disgust, etc., can be considered as a text classification problem. However, it introduces a challenging learning scenario where multiple emotions with different intensities are often found in a single sentence. Moreover, some emotions co-occur more often while other emotions rarely co-exist. In this paper, we propose a novel approach based on emotion distribution learning in order to address the aforementioned issues. The key idea is to learn a  mapping function from sentences to their emotion distributions describing multiple emotions and their respective intensities. Moreover, the relations of emotions are captured based on the Plutchik's wheel of emotions and are subsequently incorporated into the learning algorithm in order to improve the accuracy of emotion detection. Experimental results show that the proposed approach can effectively deal with the emotion distribution detection problem and perform remarkably better than both the state-of-the-art emotion detection method and multi-label learning methods.
