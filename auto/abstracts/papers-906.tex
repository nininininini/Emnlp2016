Natural language acquisition relies on appropriate generalization: the ability to produce novel sentences, while learning to restrict productions to acceptable forms in the language. Psycholinguists have proposed various properties that might play a role in guiding appropriate generalizations, looking at learning of verb alternations as a testbed. Several computational cognitive models have explored aspects of this phenomenon, but their results are hard to compare given the high variability in the linguistic properties represented in their input. In this paper, we directly compare two recent approaches, a Bayesian model and a connectionist model, in their ability to replicate human judgments of appropriate generalizations. We find that the Bayesian model more accurately mimics the judgments due to its richer learning mechanism that can exploit distributional properties of the input in a manner consistent with human behaviour.
