Understanding expression of emotions in support forums has great value and NLP methods are key to automating this. Many approaches use subjective categories which are more fine-grained than a straightforward polarity-based spectrum. However, the definition of such categories is non-trivial, and we argue for a need to incorporate communicative elements even beyond subjectivity. To support our position, we report experiments on a sentiment-labelled corpus of posts from a medical support forum. We argue that a more fine-grained approach to text analysis important, and also simultaneously recognising the social function behind affective expressions enables a more accurate and valuable level of understanding.
