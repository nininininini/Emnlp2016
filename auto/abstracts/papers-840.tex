Words shift in meaning for many reasons, including cultural factors like new technologies and regular linguistic processes like subjectification. Understanding the evolution of language and culture requires disentangling these underlying causes. Here we show how two different distributional measures can be used to detect two different types of semantic change. The first measure, which has been used in many previous works, analyzes global shifts in a word's distributional semantics; it is sensitive to changes due to regular processes of linguistic drift, such as the semantic generalization of promise (``I promise.'' -> ``It promised to be exciting.''). The second measure, which we develop here, focuses on local changes to a word's nearest semantic neighbors; it is more sensitive to cultural shifts, such as the change in the meaning of cell (``prison cell'' ->``cell phone''). Comparing measurements made by these two methods allows researchers to determine whether changes are more cultural or linguistic in nature, a distinction that is essential to work in the digital humanities and historical linguistics.
