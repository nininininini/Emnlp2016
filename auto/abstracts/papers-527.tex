Most of the state-of-the-art sentiment classification methods are based on supervised learning algorithms which require large amounts of manually labeled data. However, the labeled resources are usually imbalanced in different languages. Cross-lingual sentiment classification tackles the problem by adapting the sentiment resources in a resource-rich language to resource-poor languages. In this study, we propose an attention-based bilingual representation learning model which learns the distributed semantics of the documents in both the source and the target languages. In each language, we use Long Short Term Memory (LSTM) network to model the documents, which has been proved to be very effective for word sequences. Meanwhile, we propose a hierarchical attention mechanism for the bilingual LSTM network. The sentence-level attention model learns which sentences of a document are more important for determining the overall sentiment while the word-level attention model learns which words in each sentence are decisive. The proposed model achieves good results on a benchmark dataset using English as the source language and Chinese as the target language.
