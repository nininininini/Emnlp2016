Recently, a number of deep-learning based models have been proposed for the task of Visual Question Answering (VQA). The performance of most models is clustered around 60-70\%. In this paper we propose systematic methods to analyze the behavior of these models as a first step towards recognizing their strengths and weaknesses, and identifying the most fruitful directions for progress. We analyze two models, one each from two major classes of VQA models -- with-attention and without-attention and show the similarities and differences in the behavior of these models. We also analyze the winning entry of the VQA Challenge 2016. Our behavior analysis reveals that despite recent progress, today's VQA models are ``myopic'' (tend to fail on sufficiently novel instances), often ``jump to conclusions'' (converge on a predicted answer after 'listening' to just half the question), and are ``stubborn'' (do not change their answers across images).
