Entities and events in the world have no frequency, but our communication about them and the expressions we use to refer to them do have a strong frequency profile. Language expressions and their meanings follow a Zipfian distribution, featuring a small amount of very frequent observations and a very long tail of low frequent observations. Since our NLP datasets sample texts but do not sample the world, they are no exception to Zipf's law. This causes a lack of representativeness in our NLP tasks, leading to models that can capture the head phenomena in language, but fail when dealing with the long tail. We therefore propose a referential challenge for semantic NLP that reflects a higher degree of ambiguity and variance and captures a large range of small real-world phenomena. To perform well, systems would have to show deep understanding on the linguistic tail.
