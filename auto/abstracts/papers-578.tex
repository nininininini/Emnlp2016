Users prefer natural language software requirements because of their usability and accessibility. When they describe their wishes for software development, they often provide off-topic information. We therefore present REaCT, an automated approach for identifying and semantically annotating the on-topic parts of requirement descriptions. It is designed to support requirement engineers in the elicitation process on detecting and analyzing requirements in user-generated content. Since no lexical resources with domain-specific information about requirements are available, we created a corpus of requirements written in controlled language by instructed users and uncontrolled language by uninstructed users. We annotated these requirements regarding predicate-argument structures, conditions, priorities, motivations and semantic roles and used this information to train classifiers for information extraction purposes. REaCT achieves an accuracy of 92\% for the on- and off-topic classification task and an F1-measure of 72\% for the semantic annotation.
