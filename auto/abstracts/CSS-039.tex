Anxiety has a special importance in politics since the emotion is tied to decision-making under uncertainty, a feature of democratic institutions.  Yet, measuring specific emotions like anxiety in political settings remains a challenging task.  The present study tackles this problem by making use of natural language processing (NLP) tools to detect anxiety in a corpus of digitized parliamentary debates from Canada.  I rely upon a vector space model to rank parliamentary speeches based on the semantic similarity of their words and syntax with a set of common expressions of anxiety.  After assessing the performance of this approach with annotated corpora, I use it to test an implementation of state-trait anxiety theory.  The findings support the hypothesis that political issues with a lower degree of familiarity, such as foreign affairs and immigration, are more anxiogenic than average, a conclusion that appears robust to estimators accounting for unobserved individual traits.
