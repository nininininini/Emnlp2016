Pinyin is the most widely used romanization scheme for Mandarin Chinese. We consider the task of language identification in Pinyin-English codeswitched texts, a task that is significant because of its application to codeswitched text input. We create a codeswitched corpus by extracting and automatically labeling existing Mandarin-English codeswitched corpora. On language identification, we find that SVM produces the best result when using word-level segmentation, achieving 99.3\% F1 on a Weibo dataset, while a linear-chain CRF produces the best result at the letter level, achieving 98.2\% F1. We then pass the output of our models to a system that converts Pinyin back to Chinese characters to simulate codeswitched text input. Our method achieves the same level of performance as an oracle system that has perfect knowledge of token-level language identity. This result demonstrates that Pinyin identification is not the bottleneck towards developing a Chinese-English codeswitched Input Method Editor, and future work should focus on the Pinyin-to-Chinese character conversion step.
