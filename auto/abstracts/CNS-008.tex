In this paper, we investigate the distribution of narrative schemas (Chambers and Jurafsky, 2009) throughout different document categories and how the structure of narrative schemas is conditioned by document category, the converse of the relationship explored in Simonson and Davis (2015). We evaluate cross-category narrative differences by assessing the predictability of verbs in each category and the salience of arguments to events that narrative schemas highlight. For the former, we use the narrative cloze task employed in previous work on schemas. For the latter, we introduce a task that employs narrative schemas called narrative argument salience through entities annotated, or NASTEA. We compare the schemas induced from the entire corpus to those from the subcorpora for each topic using these two types of evaluation. Results of each evaluation vary by each topical subcorpus, in some cases showing improvement, but the NASTEA task additionally reveals that some the documents within some topics are significantly more rigid in their narrative structure, instantiating a limited number of schemas in a highly predictable fashion.
