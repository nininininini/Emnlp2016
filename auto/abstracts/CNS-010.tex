The style of narrative news affects how it is interpreted and received by readers. Two key stylistic characteristics of narrative text are point of view and diegesis: respectively, whether the narrative recounts events person- ally or impersonally, and whether the narrator is involved in the events of the story. Although central to the interpretation and reception of news, and of narratives more generally, there has been no prior work on automatically iden- tifying these two characteristics in text. We develop automatic classifiers for point of view and diegesis, and compare the performance of different feature sets for both. We built a gold- standard corpus where we double-annotated to substantial agreement (κ > 0.59) 270 En- glish novels for point of view and diegesis. As might be expected, personal pronouns com- prise the best features for point of view clas- sification, achieving an average F1 of 0.928. For diegesis, the best features were personal pronouns and the occurrences of first person pronouns in the argument of verbs, achieving an average F1 of 0.898. We apply the clas- sifier to nearly 40,000 news texts across five different corpora comprising multiple genres (including newswire, opinion, blog posts, and scientific press releases), and show that the point of view and diegesis correlates largely as expected with the nominal genre of the texts. We release the training data and the classifier for use by the community.
