Directly reading documents and being able to answer questions from them is an unsolved challenge. To avoid its inherent difficulty, question answering (QA) has been directed towards using Knowledge Bases (KBs) instead, which has proven effective. Unfortunately KBs often suffer from being too restrictive, as the schema cannot support certain types of answers, and too sparse, e.g. Wikipedia contains much more information than Freebase. In this work we introduce a new method, Key-Value Memory Networks, that makes reading documents more viable by utilizing different encodings in the addressing and output stages of the memory read operation. To compare using KBs, information extraction or Wikipedia documents directly in a single framework we construct an analysis tool, WikiMovies, a QA dataset that contains raw text alongside a preprocessed KB, in the domain of movies. Our method reduces the gap between all three settings. It also achieves state-of-the-art results on the existing WikiQA benchmark.
