Language documentation begins by gathering speech. Manual or automatic transcription at the word level is typically not possible because of the absence of an orthography or prior lexicon, and though manual phonemic transcription is possible, it is very slow. On the other hand, translations of the minority language into a major language are more easily acquired. We propose a method to harness such translations to improve automatic phoneme recognition. The method assumes no prior lexicon or translation model, instead learning them from phoneme lattices and translations of the speech being transcribed. Experiments demonstrate phoneme error rate improvements against two baselines and the model's ability to learn useful bilingual lexical entries.
