Stance detection is the task of classifying the attitude  expressed in a text towards a target such as Hillary Clinton to be ''positive'', ''negative'' or ''neutral''. Previous work has assumed that either the target is mentioned in the text or that training data for every target is given. This paper considers the more challenging version of this task, where targets are not always mentioned and no training data is available for the test targets. We experiment with conditional LSTM encoding, which builds a representation of the tweet that is dependent on the target, and demonstrate that it outperforms encoding the tweet and the target independently. Performance is improved further when the conditional model is augmented with bidirectional encoding. We evaluate our approach on the SemEval 2016 Task 6 Twitter Stance Detection corpus achieving performance second best only to a system trained on semi-automatically labelled tweets for the test target. When such weak supervision is added, our approach achieves state-of-the-art results.
