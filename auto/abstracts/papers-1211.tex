Psychological analysis of language has repeatedly shown that an individual's rate of mentioning 1st person singular pronouns predicts a wealth of important demographic and psychological factors. However, these analyses are performed out of context --- syntactic and semantic --- which may change the magnitude or even direction of such relationships.  In this paper, we put ``pronouns in their context'', exploring the relationship between self-reference and age, gender, and depression depending on syntactic position and verbal governor. We find that pronouns are overall more predictive when taking dependency relations and verb semantic categories into account, and, the direction of the relationship can change depending on the semantic class of the verbal governor.
